
%\documentclass[useAMS, usenatbib, fleqn]{mn2e}

% PRD specific
\documentclass[aps,prd,showpacs,superscriptaddress,groupedaddress]{revtex4}  % twocolumn submission
%\documentclass[aps,preprint,showpacs,superscriptaddress,groupedaddress]{revtex4}  % for double-spaced preprint
\usepackage{dcolumn}   % needed for some tables
\usepackage{bm}        % for math
% avoids incorrect hyphenation, added Nov/08 by SSR
\hyphenation{ALPGEN}
\hyphenation{EVTGEN}
\hyphenation{PYTHIA}

\usepackage{microtype}
\usepackage{aas_macros}
\usepackage{times}
%\usepackage{txfonts}
\usepackage{url}
\usepackage{amsmath}
\usepackage{amsbsy}
\usepackage{graphicx}
\usepackage{subfig}
\usepackage{xspace}
\usepackage{float}
\usepackage{caption}
\usepackage{amsfonts}
\usepackage{amssymb}
\usepackage{multirow}
\usepackage{color}	
\usepackage[breaklinks, colorlinks, citecolor=blue, linkcolor=black, urlcolor=black]{hyperref}%with colors
%\usepackage{hyperref}%without colors



\newcommand{\ie}{{{i.e.}~}}
\newcommand{\eg}{{{e.g.}~}}
\newcommand{\equref}[1]{{\xspace}Eq.~(\ref{#1})}
\newcommand{\figref}[1]{{\xspace}Fig.~\ref{#1}}
\newcommand{\equrefbegin}[1]{{\xspace}Equation~(\ref{#1})}
\newcommand{\figrefbegin}[1]{{\xspace}Figure~\ref{#1}}
\newcommand{\secref}[1]{{\xspace}Sec.~\ref{#1}}
\renewcommand{\d}{{\mathrm{d}}}
\newcommand{\equ}[1]{\begin{equation}#1\end{equation}}
\newcommand{\eqn}[1]{\begin{eqnarray}#1\end{eqnarray}}
\renewcommand{\vec}[1]{\bmath{#1}}
\newcommand{\negsp}[1]{\hspace*{-#1mm}}

\newcommand{\gal}{g}
\newcommand{\nobj}{{N_{\rm gal}}}
\newcommand{\band}{b}

\newcommand{\todo}[1]{\textcolor{blue}{[TODO: #1]}}
\newcommand{\bl}[1]{\textcolor{blue}{[BL: #1]}}
\newcommand{\dwh}[1]{\textcolor{cyan}{[DWH: #1]}}

\sloppy\sloppypar\raggedbottom\frenchspacing
\begin{document}

 
\title{Shrinking stellar parallaxes with color-magnitude information but no use of stellar models}

\author{Boris~Leistedt}
  \email{boris.leistedt@nyu.edu}
  \affiliation{Center for Cosmology and Particle Physics, Department of Physics, New York University, New York, NY 10003, USA}
  \affiliation{NASA Einstein Fellow}
   
\author{David~W.~Hogg}
  \email{david.hogg@nyu.edu}
  \affiliation{Center for Cosmology and Particle Physics, Department of Physics, New York University, New York, NY 10003, USA}
  \affiliation{Center for Data Science, New York University, 60 Fifth Avenue, New York, NY 10011, USA}
  \affiliation{Flatiron Institute, 162 Fifth Avenue, New York, NY 10010, USA}
  
\begin{abstract}
We present a hierarchical probabilistic model for obtaining improved stellar distances estimates with both multicolor and parallax information. 
This is achieved with a data driven model of the color--magnitude diagram, not relying on stellar models but instead on the  relative abundances of stars in color--magnitude cells.
The latter are inferred from noisy magnitudes and parallaxes; observational errors are deconvolved into a noiseless color--magnitude diagram, which can be useful for a range of applications.
Here, we leverage multicolor information to provide narrower posterior distributions on stellar distances, a process which is particularly efficient in the dense regions of the color--magnitude diagram.
We demonstrate the power of this approach on the Gaia TGAS data sample with APASS magnitudes.
We validate the method on a random subsample and also on open clusters.
We find that the number of objects with signal-to-noise ratio lower than 20 is halved, and that the distance errors are also halved for 20\% of objects.
We make our improved distance estimates publicly available.
\end{abstract}

\pacs{TODO}

\maketitle

  
%%%%%%%%%%%%%%%%%%%%%%%%%%%%%%%%%%%%%%
\section{Introduction}


%%%%%%%%%%%%%%%%%%%%%%%%%%%%%%%%%%%%%%
\section{Model}


\begin{table} %%%%
\centering
\begin{tabular}{cl}
\hline
$s$	&	index of object \\
$b$	&	index of color--magnitude bin\\
$b_s$	&	index of color--magnitude bin\\
$n_b$	& 	galaxy count in the $b$-th color--magnitude bin  \\
$\{n_b\}$	&	set of all galaxy counts $n_b$, summing to $\nobj$\\
$f_b$	&	fractional galaxy count in the $b$-th color--magnitude bin  \\
$\{f_b\}$	&	set of all fractional bin counts $f_b$, summing to $1$\\
$M_s, \{C_j\}_s$	&	properties of the $s$-th stars	\\
$\{ d_s, b_s\}$	&	set of properties of all stars in the sample	\\
$\hat{m}_s, \hat{C}_s$ 	&	\\
$\{ \hat{m}_s, \hat{C}_s \}$ &	\\
$\{ d_s, b_s, \hat{m}_s, \hat{C}_s, \hat{\varpi}_s \}$	 	&	\\
\hline
\end{tabular}
\caption{Summary of our notation. }
\label{tab:notation}
\end{table} 



\equ{
	p\left(\hat{\varpi}_s \bigr\rvert d_s\right) = \mathcal{N}\bigl(\hat{\varpi}_s - d_s^{-1};\sigma_{\hat{\varpi}_s}^2 \bigr)
}

\equ{
	p\left(\hat{m}_s, \hat{C}_s \bigr\rvert M_s, d_s, C_s\right)  =  \mathcal{N}\bigl( M_s + 5\log_{10}d_s  -\hat{m}_s ;\sigma_{\hat{m}_s}^2 \bigr) \  \mathcal{N}\bigl(\hat{C}_s - C_s;\sigma_{\hat{C}_s}^2 \bigr)
}

\eqn{
	p\left(b_s \bigr\rvert \bigl\{ f_b \bigr\}\right) &=& f_{b_s} \\ 
	p\left(M_s, C_s \bigr\rvert b_s \right) &=& \mathcal{N}\bigl(M_s - \mu_{b,0};\sigma_{b,0}^2 \bigr)  \ \mathcal{N}\bigl(C_s - \mu_{b,1};\sigma_{b,1}^2 \bigr)
}


\eqn{
	p\left(\bigl\{ f_b \bigr\} \bigr\rvert \bigl\{ d_s, b_s, \hat{m}_s, \hat{C}_s, \hat{\varpi}_s \bigr\} \right) \propto p\bigl( \bigl\{ f_b \bigr\} \bigr\rvert \{n_b \} \bigr) \propto \prod_b \frac{ f_b^{n_b} }{n_b !}
}


\eqn{
	p\left(d_s \bigr\rvert \bigl\{ f_b \bigr\}, b_s, \hat{m}_s, \hat{C}_s, \hat{\varpi}_s\right) &\propto& f_{b_s} \ \mathcal{N}\bigl(\hat{\varpi}_s - d_s^{-1};\sigma_{\hat{\varpi}_s}^2 \bigr) \  \mathcal{N}\bigl( \mu_{b,0} + 5\log_{10}d_s  -\hat{m}_s ;\sigma_{\hat{m}_s}^2 + \sigma_{b,0}^2 \bigr) 
}


\eqn{
	p\left(b_s \bigr\rvert \bigl\{ f_b \bigr\}, d_s, \hat{m}_s, \hat{C}_s, \hat{\varpi}_s\right) &\propto& f_{b_s} \  \mathcal{N}\bigl( \mu_{b_s,0} + 5\log_{10}d_s  -\hat{m}_s ;\sigma_{\hat{m}_s}^2 + \sigma_{b_s,0}^2 \bigr) \  \mathcal{N}\bigl(\hat{C}_s - \mu_{b_s,1};\sigma_{\hat{C}_s}^2 + \sigma_{b_s,1}^2 \bigr)
}

%%%%%%%%%%%%%%%%%%%%%%%%%%%%%%%%%%%%%%
\section{Application to TGAS}

\begin{figure}
\caption{Parallax and color errors for training and validation sets}
\end{figure}

\begin{figure}
\caption{3 Noisy HRD with errors or not? 3 deconvolved HRD with error maps. 3 deconvolved HRD with error maps. }
\end{figure}

\begin{figure}
\caption{Validation parallaxes : scatter plot of distances without and with shrinkage}
\end{figure}

\begin{figure}
\caption{Show a few PDFs, including some with basically no parallax measurements}
\end{figure}


%%%%%%%%%%%%%%%%%%%%%%%%%%%%%%%%%%%%%%
\section{Acknowledgements}

We thank Andrew Casey for useful conversations.

This project was developed in part at the 2016 NYC Gaia Sprint, hosted by the Center for Computational Astrophysics at the Simons Foundation in New York City.

This work has made use of data from the European Space Agency (ESA) mission Gaia (http://www.cosmos.esa.int/gaia), processed by the Gaia Data Processing and Analysis Consortium (DPAC, http://www.cosmos.esa.int/web/gaia/dpac/consortium). Funding for the DPAC has been provided by national institutions, in particular the institutions participating in the Gaia Multilateral Agreement.

BL was supported by NASA through the Einstein Postdoctoral Fellowship (award number PF6-170154).
DWH was partially supported by the NSF (AST-1517237) and the Moore--Sloan Data Science Environment at NYU.


%%%%%%%%%%%%%%%%%%%%%%%%%%%%%%%%%%%%%%
\bibliography{bib}

%%%%%%%%%%%%%%%%%%%%%%%%%%%%%%%%%%%%%%
\appendix

%% %%%%%%%%%%%%%%%%%%%%%%%%%%%%%%%%%%%%
\end{document}
%% %%%%%%%%%%%%%%%%%%%%%%%%%%%%%%%%%%%%
